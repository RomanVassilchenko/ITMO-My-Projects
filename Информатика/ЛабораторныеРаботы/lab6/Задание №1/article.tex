\documentclass[main.tex]{subfiles}

\begin{document}
\begin{center}
\includegraphics[width=1\textwidth]{picture}
\end{center}

\begin{multicols}{2}
\begin{flushleft}Ф. Бартенев\end{flushleft}

\begin{flushleft}
\huge{\textbf{Наблюдения\\*в математике}} \par
\end{flushleft}

\begin{addmargin}[3em]{0em}
{\small \textit{Свойства чисел, известные сегодня, по большей части были открыты путем наблюдений.} \par}
\end{addmargin}

\begin{flushright}{\small \emph{Л е о н а р д  Э й л е р}\par}\end{flushright}

\noindent Как обычно исследуются законы природы?

Физики, химики, биологи ставят эксперименты.	Затем исследуют получающиеся результаты измерений, подмечают закономерности, делают те или иные выводы.

Как ни удивительно, но и ученые-математики в своем научном творчестве зачастую пользуются наблюдениями, как и естествоиспытатели. Чтобы глубоко понимать даже ``школьную'' математику, решать различные замысловатые задачи, нужно научиться наблюдать в математике.

Вот один интересный факт истории науки. Выдающимся математикам К. Гауссу и А. Колмогорову в детстве предложили аналогичные задачи: десятилетнему Гауссу - найти сумму $1 + 2 + 3 + \dots + 98 + 99 + 100$,  а шестилетнему Колмогорову - сумму $1 + 3 + 5 + \dots + 95 + 97 + 99$. Гаусс решил свою задачу почти мгновенно заметив, что $1 + 99 = 2 + 98 = 3 + 97 = \dots$\\*Колмогоров, посчитав суммы $1 + 3, 1 + 3 + 5, 1 + 3 + 5 + 7$, заметил, что $1 = 1^{2}, 1 + 3 = 2^{2}, 1 + 3 + 5 = 3^{2}, 1 + 3 + 5 + 7 = 4^{2}$, и также быстро нашел ответ.

А как решить такую задачу:

\textit{Найти две последние цифры числа $311^{28}$?}

Не вычислять же в самом деле 28-ю степенб числа 311? Конечно нет!  Достаточно заметить, что последние цифры числа $311^{1}$ - это 11, числа $311^{2}$ - 21, $311^{3}$ - 31, $\dots$; так что у числа $311^{28}$ две последние цифры это 81

Рассмотрим еще одну похожую задачу:

\textit{Какими двумя цифрами оканчивается число $7^{51}$?}

Найдем вначале две последние цифры у чисел $7^{1}, 7^{2}, 7^{3}, 7^{4} и 7^{5}$. Легко видеть, что у $7^{1}$ всего одна цифра - 7; запишем его так: 07; у $7^{2}$ - 49, у $7^{3}$ - 41 (две последние цифры произведения 49 на 7), у $7^{4}$ - 01, так что у $7^{5}$ две последние цифры снова 07. Мы видим, что две последние цифры ступеней семерки чередуются образуя так называемую \textit{переодическую последовательность} с периодом 4:
\\*
\underline{07, 49, 43, 01,} \underline{07, 49, 43, 01,} \underline{07, 49, ...}

Остается только найти остаток от деления числа 51 на 4 - он равен трем, и посмотреть, какими двумя цифрами оканчивается куб семерки: у чисел $7^{51} и 7^{3} две последние цифры совпадают$. Итак, вот \emph{ответ} : две последние цифры числа $7^{51}$ - это 43.

Решим еще одну задачу, отличную от превыдущих.

\textit{Для каких чисел справедливо неравенство:}

$\|a| + |b| > |a + b|?$

Рассмотрим несколько примеров.
Для этого заполним несколько строк следующей таблицы:
\begin{center}
\begin{tabular}{ |c|c|c|c| } 
\hline
 a & b & |a| + |b| & |a + b| \\ [0.5ex] 
 \hline
 -3  & 2 & 5 & 1 \\ 
 -7  & -1 & 8 & 8 \\ 
 0 & -4 & 4 & 4 \\ 
  4 & 0 & 4 & 4 \\ 
   2 & 3 & 5 & 5 \\ 
    7 & -1 & 8 & 6 \\ 
    ... & ... & ... & ... \\ [0.7ex]
 \hline
\end{tabular}
\end{center}
Глядя на таблицу, уже нетрудно догадаться, каков должен быть ответ в последней задаче: сумма абсолютных величин двух чисел больше абсолютной величины суммы этих чисел только тогда, когда эти числа отличны от нуля и имеют разные знаки.

Конечно же, всякая ``догадка'', гипотеза, нуждается в строгом математическом доказательстве. Это мы оставляем нашим читателям.

Если вы разобрались в решении первых трех задач, попробуйте решить самостоятельно еще три задачи немного посложнее.

{footnotesize  Задачи

 1.Какими цифрами не могут оканчиваться суммы $1 + 2, 1 + 2 + 3, 1 + 2 + 3 + 4, 1 + 2 + 3 + 4 + \dots$?\\
 2. По кругу расположены 60 положительных и отрицательных единиц (причем есть и те и другие). Известно, что произведение любых трех последовательных чисел равно -1. Найдите сумму всех чисел.\\
 3. Сколько потребуется разрезов, чтобы куб со стороной 3 см разрезать на кубики со стороной 1 см?
\par}
 
 В заключении предостережем читателей от скоропалительных выводов из результатов наблюдений. Чтобы доказать какое-либо утверждение, нужно убедиться, что оно справедливо в каждом частном случае; в тоже время, чтобы \textit{опровергнуть} какое-либо утверждение, достаточно показать, что оно не справедливо хотя бы в одном частном случае (как говорят математики, привести \textit{контрпример}). Поясним, что мы имеет ввиду.
 
 Рассмотрим множество чисел вида $m = n^{2} + 3n + 1$, где n - любое натуральное число. Для значений $n = 1, 2 , 3, 4 и 5$ получаем $m = 5, 11, 19, 29, 41$, то есть первые пять значений m - простые числа. Можно ли сделать вывод, что любой элемент нашего множества будет простым числом (для любого натурального n)? Разумеется, нет: уже при $n = 6$ получаем составное значение $m = 55$
 
{footnotesize  Задачи

 4. Пусть А - множество чисел вида $6n - 1$, а P - множество простых чисел. Верно или ложно высказывание: $A \subset P$, если $n \in N$?

 5. Верно ли, что при любом целом n справедливо неравенство $4n^{2} + 40n + 99 > 0$?
 \par}
 
 Иногда найти контрпример, опровергающий общее утверждение, очень трудно. Бывает, что легче доказать \textit{существование} соответствующего контрпримера, чем его \textit{построить}.
 
 Например, очевидно, что если 100 школьников получили 101 тетрадь, причем каждый школьник получил хотя бы одну тетрадь, то найдется школьник, получивший две тетради.
 
Прием рассуждений, с помощью которого мы сделали вывод в последнем примере, называется \textit{``принципом Дирихле''} и часто применяется к решению задач. Решим с его помощью следующую задачу:

 \textit{Доказать, что из 101 числа можно выбрать два, разность которых делится на 100}.
 
 В самом деле, при делении на 100 в остатке может получиться одно из следующих чисел: $0, 1, 2, \dots, 99$ - сто разных чисел. Поэтому среди 101 числа обязательно найдутся два, дающие при делении на 100 равные остатки. Следовательно, их разность делится на 100.
 
 И, наконец, еще одна задача на принцип Дирихле.
 
{\footnotesize Задача 6. В лесу растет 710 000 елочек. На каждой елочке не больше 100 000 иголочек. Докажите, что в лесу есть по крайней мере 8 елочек с одинаковым числом иголочек. \par}
\end{multicols}

\end{document}